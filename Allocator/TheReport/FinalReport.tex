%%%%%%%%%%%%%%%%%%%%%%%%%%%%%%%%%%%%%%%%%
% Programming/Coding Assignment
% LaTeX Template
%
% This template has been downloaded from:
% http://www.latextemplates.com
%
% Original author:
% Ted Pavlic (http://www.tedpavlic.com)
%
% Note:
% The \lipsum[#] commands throughout this template generate dummy text
% to fill the template out. These commands should all be removed when 
% writing assignment content.
%
% This template uses a Perl script as an example snippet of code, most other
% languages are also usable. Configure them in the "CODE INCLUSION 
% CONFIGURATION" section.
%
%%%%%%%%%%%%%%%%%%%%%%%%%%%%%%%%%%%%%%%%%

%----------------------------------------------------------------------------------------
%	PACKAGES AND OTHER DOCUMENT CONFIGURATIONS
%----------------------------------------------------------------------------------------

\documentclass{article}

\usepackage{fancyhdr} % Required for custom headers
\usepackage{lastpage} % Required to determine the last page for the footer
\usepackage{extramarks} % Required for headers and footers
\usepackage[usenames,dvipsnames]{color} % Required for custom colors
\usepackage{graphicx} % Required to insert images
\usepackage{listings} % Required for insertion of code
\usepackage{courier} % Required for the courier font
\usepackage{pgfgantt}

% Margins
\topmargin=-0.45in
\evensidemargin=0in
\oddsidemargin=0in
\textwidth=6.5in
\textheight=9.0in
\headsep=0.3in

\linespread{1.1} % Line spacing

% Set up the header and footer
\pagestyle{fancy}
\lhead{\hmwkAuthorName} % Top left header
\chead{\hmwkTitle} % Top center head
\rhead{\hmwkClass } % Top right header
\lfoot{} % Bottom left footer
\cfoot{} % Bottom center footer
\rfoot{Page\ \thepage\ of\ \protect\pageref{LastPage}} % Bottom right footer
\renewcommand\headrulewidth{0.4pt} % Size of the header rule
\renewcommand\footrulewidth{0.4pt} % Size of the footer rule

\setlength\parindent{0pt} % Removes all indentation from paragraphs

%----------------------------------------------------------------------------------------
%	CODE INCLUSION CONFIGURATION
%----------------------------------------------------------------------------------------

\definecolor{MyDarkGreen}{rgb}{0.0,0.4,0.0} % This is the color used for comments
\lstloadlanguages{Java} % Load Java syntax for listings, for a list of other languages supported see: ftp://ftp.tex.ac.uk/tex-archive/macros/latex/contrib/listings/listings.pdf
\lstset{language=Java, % Use Java in this example
        frame=none, % Single frame around code
        basicstyle=\small\ttfamily, % Use small true type font
        keywordstyle=[1]\color{Blue}\bf, % Perl functions bold and blue
        keywordstyle=[2]\color{Purple}, % Perl function arguments purple
        keywordstyle=[3]\color{Blue}\underbar, % Custom functions underlined and blue
        identifierstyle=, % Nothing special about identifiers                                         
        commentstyle=\usefont{T1}{pcr}{m}{sl}\color{MyDarkGreen}\small, % Comments small dark green courier font
        stringstyle=\color{Purple}, % Strings are purple
        showstringspaces=false, % Don't put marks in string spaces
        tabsize=8, % 5 spaces per tab
        %
        % Put standard Perl functions not included in the default language here
        morekeywords={rand},
        %
        % Put Perl function parameters here
        morekeywords=[2]{on, off, interp},
        %
        % Put user defined functions here
        morekeywords=[3]{test},
       	%
        morecomment=[l][\color{Blue}]{...}, % Line continuation (...) like blue comment
        numbers=left, % Line numbers on left
        firstnumber=1, % Line numbers start with line 1
        numberstyle=\tiny\color{Blue}, % Line numbers are blue and small
        stepnumber=100 % Line numbers go in steps of 5
}


%----------------------------------------------------------------------------------------
%	DOCUMENT STRUCTURE COMMANDS
%	Skip this unless you know what you're doing
%----------------------------------------------------------------------------------------

% Header and footer for when a page split occurs within a problem environment
\newcommand{\enterProblemHeader}[1]{
\nobreak\extramarks{#1}{#1}\nobreak
\nobreak\extramarks{#1}{#1}\nobreak
}

% Header and footer for when a page split occurs between problem environments
\newcommand{\exitProblemHeader}[1]{
\nobreak\extramarks{#1}{#1 continued on next page\ldots}\nobreak
\nobreak\extramarks{#1}{}\nobreak
}

\setcounter{secnumdepth}{0} % Removes default section numbers
\newcounter{homeworkProblemCounter} % Creates a counter to keep track of the number of problems

\newcommand{\homeworkProblemName}{}
\newenvironment{homeworkProblem}[1][
 \arabic{homeworkProblemCounter}]{ % Makes a new environment called homeworkProblem which takes 1 argument (custom name) but the default is "Problem #"
\stepcounter{homeworkProblemCounter} % Increase counter for number of problems
\renewcommand{\homeworkProblemName}{#1} % Assign \homeworkProblemName the name of the problem
\section{\homeworkProblemName} % Make a section in the document with the custom problem count
\enterProblemHeader{} % Header and footer within the environment
}{
\exitProblemHeader{} % Header and footer after the environment
}

\newcommand{\problemAnswer}[1]{ % Defines the problem answer command with the content as the only argument
\noindent\framebox[\columnwidth][c]{\begin{minipage}{0.98\columnwidth}#1\end{minipage}} % Makes the box around the problem answer and puts the content inside
}

\newcommand{\homeworkSectionName}{}
\newenvironment{homeworkSection}[1]{ % New environment for sections within homework problems, takes 1 argument - the name of the section
\renewcommand{\homeworkSectionName}{#1} % Assign \homeworkSectionName to the name of the section from the environment argument
\subsection{\homeworkSectionName} % Make a subsection with the custom name of the subsection
\enterProblemHeader{\homeworkProblemName\ [\homeworkSectionName]} % Header and footer within the environment
}{
\enterProblemHeader{\homeworkProblemName} % Header and footer after the environment
}


%----------------------------------------------------------------------------------------
%	NAME AND CLASS SECTION
%----------------------------------------------------------------------------------------

\newcommand{\hmwkTitle}{Final Report} % Assignment title
\newcommand{\hmwkDueDate}{29th April 2016} % Due date
\newcommand{\hmwkClass}{COMP30050} % Course/class
\newcommand{\hmwkClassTime}{Software Engineering Project 3} % Class/lecture time
\newcommand{\hmwkClassInstructor}{Dr. Tony Veale} % Teacher/lecturer
\newcommand{\hmwkAuthorName}{Lamp} % Your name

%----------------------------------------------------------------------------------------
%	TITLE PAGE
%----------------------------------------------------------------------------------------

\title{
\vspace{2in}
\textmd{\textbf{\hmwkClass:\ \hmwkClassTime}}\\
\normalsize\
\vspace{0.1in}\large{\textit{\hmwkClassInstructor}}\\
\vspace{0.2in}
\textmd{\textbf{\hmwkTitle}}\\
\small{Due\ on\ \hmwkDueDate}\\
\vspace{.5in}
}

\author{\textbf{\hmwkAuthorName}\\
Joe Duffin - 13738019\\
Edwin Keville - 13718661\\
Niamh Kavanagh - 12495522\\
Gerard Fogarty - 13303911 (\textit{the lost lamp})
}

\date{} % Insert date here if you want it to appear below your name

%----------------------------------------------------------------------------------------

\begin{document}
\begin{titlepage}
\maketitle
\thispagestyle{empty}
\end{titlepage}

\newpage
%----------------------------------------------------------------------------------------
%	TABLE OF CONTENTS
%----------------------------------------------------------------------------------------

%\setcounter{tocdepth}{1} % Uncomment this line if you don't want subsections listed in the ToC

\newpage
\tableofcontents
\newpage

%----------------------------------------------------------------------------------------

\begin{homeworkProblem}[Introduction]



\begin{homeworkSection}{Hi}
hi hi hi
\end{homeworkSection}

\begin{homeworkSection}{Bye}
Bye bye bye
\end{homeworkSection}








\end{homeworkProblem}
\newpage

%----------------------------------------------------------------------------------------

\begin{homeworkProblem}[What we produced/Interfaces]

One Gui and 3 command line interfaces. 
We wanted a fine mix of an ultimate solution, but also lots of modularity and re-usability.

\begin{homeworkSection}{The Graphical User Interface}
The hybrid solution, the actual submission.
\end{homeworkSection}

\begin{homeworkSection}{The Command Line Version}
This provides an all the functionality of the gui and more. By default it presents a hybrid solution.
\end{homeworkSection}

\begin{homeworkSection}{An individual and Complete Genetic Solution Command Line}
Loads of functionality eg inputs/outputs
\end{homeworkSection}

\begin{homeworkSection}{An individual and Complete Annealed Solution Command Line}
Loads of functionality eg inputs/outputs
\end{homeworkSection}








\end{homeworkProblem}
\newpage

%------------------------------------------------------------------------------------------
\begin{homeworkProblem}[Performance Analysis]
Scripted overnight run to attain these figures and graphs

\begin{homeworkSection}{The Genetic Alogrithm}
There Parameters: abc exists.\\
Ran n times with these versions of abc.\\
1 Graph ideally. excel
\end{homeworkSection}


\begin{homeworkSection}{The Simulated Annealing alogrithms}
There Parameters: abc exists.\\
Ran n times with these versions of abc.\\
1 Graph ideally. excel
\end{homeworkSection}


\begin{homeworkSection}{The hybrid solution}
There Parameters: abc exists.\\
Ran n times with these versions of abc.\\
1 Graph ideally. excel
\end{homeworkSection}

\end{homeworkProblem}
\newpage
%------------------------------------------------------------------------------------------
\begin{homeworkProblem}[Development Phases/Issues]



\begin{homeworkSection}{The Genetic Algorithm}

mate was the biggun\\
initially when mating 2 solutions we choose the 'happiest' corresponding assignments from each and created a new assignment with these. Each assignment was independently better, the solution as a whole was pants as there were lots of penalties incurred.\\
We move to considering a complete child solution and used the notion of 'happier with' to determine which parent's assignment should be used.
\end{homeworkSection}

\begin{homeworkSection}{The Simulated Annealing Algorithm}
The key turning point being understanding the relationship between initial tmeperature and cooling amount. Number iterations is a function of these two.
\end{homeworkSection}

\begin{homeworkSection}{The Gui} 
The Gui was initially overcomplicated,,, after performance analysis we determined none of this was necessary. We elected for a simple LOAD GO SAVE option, with an epic title bar. It allows the user to create the ultimate solution, (or very close too) insert percentage.

\end{homeworkSection}{}







\end{homeworkProblem}
\newpage

%---------------------------------------------------------------------------------------
\begin{homeworkProblem}[Our Development Model]



\begin{homeworkSection}{Not sure what needs to go in here}

\end{homeworkSection}






\end{homeworkProblem}
\newpage

%---------------------------------------------------------------------------------------
\begin{homeworkProblem}[Technical Details Of Note]



\begin{homeworkSection}{Inheritance}


\end{homeworkSection}

\begin{homeworkSection}{Callback Listeners}

\end{homeworkSection}

\begin{homeworkSection}{Threads} 


\end{homeworkSection}{}







\end{homeworkProblem}
\newpage

%---------------------------------------------------------------------------------------
\begin{homeworkProblem}[Class Diagrams]



\begin{homeworkSection}{Nice Pictures}

\end{homeworkSection}







\end{homeworkProblem}
\newpage

%---------------------------------------------------------------------------------------

\begin{homeworkProblem}[The Team Night Out]
\begin{homeworkSection}{Burgers}
oooohhhhh yeeaaaaahh
\end{homeworkSection}
\begin{homeworkSection}{THE NIGHT MAN COMMETH}
WOOOOOOO
\end{homeworkSection}
\end{homeworkProblem}


\newpage
%----------------------------------------------------------------------------------------
\begin{homeworkProblem}[A Sad Day For Lamp]
\begin{homeworkSection}{The loss of a team member}
We suffered a tragic loss,,, Gerard has become a lost lamp
\end{homeworkSection}
\end{homeworkProblem}
\end{document}